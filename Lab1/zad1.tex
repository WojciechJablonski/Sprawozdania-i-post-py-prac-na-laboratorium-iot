\documentclass[a4paper,11pt]{article}
\usepackage[MeX]{polski}
\usepackage[cp1250]{inputenc}
\usepackage{color}
\usepackage{hyperref}
\usepackage[normalem]{ulem}
\usepackage{latexsym}
\usepackage{enumerate}
\title{\huge Moja pierwsza praca w programie latex}
\frenchspacing
\begin{document}
\maketitle
\tableofcontents
\newpage
\section{Formatowanie tekstu} 
\begin{itemize}
\huge \item  Rodzaje czcionek \cite{latex1} \newline
\texttt{Rozne czcionki} \\
\textmd{Rozne czcionki} \\
\textup{Rozne czcionki} \\
\textbf{Rozne czcionki} \\
\textsc{Rozne czcionki} \\
\huge \item  Male oraz duze czcionki \newline
\\

\normalsize Czcionka automatyczna \newline \\
\large Czcionka large  \newline \\
\Large Czcionka Large  \newline \\
\huge Czcionka huge  \newline \\ 
\small Czcionka small  \newline \\
\tiny Bardzo mala czcionka \newline \\



\huge \item  Kolorowanie tekstu \newline \large
{\color{red} Tu bedzie tekst czerwony} \\
{\color{green} Tu bedzie tekst zielony}\\
{\color{blue} Tu bedzie tekst niebieski}\\

\huge \item  Pogrubienie, kursywa, podkreslenie\newline \\ 
\large \textbf{Czcionka pogrubiona} \\
\textit{Czcionka pochylona}\\ 
\uline{Tekst podkreslony pojedyncza linia}\\
\uuline{Tekst podkreslony podwojna linia}\\

\item \huge Text do prawej do lewej i srodek\newline
\begin{flushleft}
\large Text do lewej
\end{flushleft}
\begin{center}
\large Tekst na srodku
\end{center}
\begin{flushright}
\large Tekst do prawej
\end{flushright}
\end{itemize}
\section{Formatowanie Tabel}
\begin{flushleft}

\large \item Kontener \newline \\ 
\begin{tabular}{|p{4.7cm}|} \hline
to jest text w z akapitami 
tu tez jest taki tekst a tu nic niema taki tekst przykladowy
\\ \hline
\end{tabular}

\large \item Tabela z 3 wierszami i kolumnami \newline \\
\begin{tabular}{|r|l|r|} \hline
Ala&  ma & kota \\
a & kot & ma   \\ \hline
ale & koniec & wierszyka \\
\hline 
\end{tabular}
\end{flushleft}


\section{Wzory}
Prosty wzor\\
\begin{displaymath}
c^{2}=a^{2}+b^{2}
\end{displaymath}
Trudny wzor

$$
\lim_{n \to \infty}
\sum_{k=1}^n \frac{1}{k^2}
= \frac{\pi^3}{10}
$$

\bibliography{zad1,zad1a}{}
\bibliographystyle{plain}
\nocite{latex2}
\nocite{*}

\end{document}